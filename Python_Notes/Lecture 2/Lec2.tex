\documentclass[11pt]{article}   %Needed for every document
\usepackage{geometry}           %Going to help get paper size right
\geometry{letterpaper}          %Normal paper
\usepackage{multicol}
\usepackage{graphicx,epsfig}    %Graphics packages and pictures
\usepackage{amssymb}            %Package for symbols

\usepackage{epstopdf}
\usepackage{color}
\graphicspath{ {Figures/} }
\DeclareGraphicsRule{.tif}{png}{.png}{`convert #1 `dirname #1`/`basename #1 .tif`.png}  %Need for pictures
\newcommand{\ignore}[1]{} %used to make inline comments
\definecolor{gray}{rgb}{0.5, 0.5, 0.5}
\newcommand{\gray}[1]{\colorbox{gray}{#1}}
\usepackage[utf8]{inputenc}
\usepackage[most]{tcolorbox}
\usepackage{hyperref}


\title{Lecture 2}
\author{Steven Walton\\     %Note that we don't close until after address to keep in title format.
\textit{PS 232 Computational Methods}\\
\textit{Department of Physics}\\
\textit{Embry-Riddle Aeronautical University}\\
\textit{Prescott, AZ   86301}}

\tcbset{
       frame code={}
       center title,
       left=0pt,
       right=0pt,
       top=0pt,
       bottom=0pt,
       colback=gray!70,
       colframe=white,
       width=\dimexpr\textwidth\relax,
       enlarge left by=0mm,
       boxsep=5pt,
       arc=0pt,outer arc=0pt,
       }

\begin{document}

\maketitle

\section*{Importing A Module}
Now that we have talked about modules, let's use them. One command that we are going to have to know is \gray{Import}.
What this does is tells python that you care going to use the module specified.  So for example, let's create a random array.
\begin{tcolorbox}
   import numpy                  $\#$ We want to import the module numpy for the command random
   s = numpy.random.random(5)    $\#$ Take special note to how this is called.  Do you have to do it this way?
   print s
\end{tcolorbox}


You will see here that we are outputting the value of s, which is an array of 5 random numbers between 0 and 1.
Notice the tricky structure of how we called this this random command.  We had to use all those periods.  What this means
is that we are using the module NumPy, then using the library of Random, and the command random in it. This can be
tiresome to write over and over, especially if we only want to be calling random from NumPy and nothing else.  We have
a way to deal with this, and to take it one step further.  See if you can understand the following code.
\begin{tcolorbox}
   from numpy import random as ran
   s = ran.random(5)
   print s
\end{tcolorbox}



\end{document}
