\documentclass[11pt]{article}   %Needed for every document
\usepackage[margin=0.75in]{geometry}           %Going to help get paper size right
\geometry{letterpaper}          %Normal paper
\usepackage{multicol}
\usepackage{graphicx,epsfig}    %Graphics packages and pictures
\usepackage{amssymb}            %Package for symbols

\usepackage{epstopdf}
\usepackage{color}
\graphicspath{ {Figures/} }
\DeclareGraphicsRule{.tif}{png}{.png}{`convert #1 `dirname #1`/`basename #1 .tif`.png}  %Need for pictures
\newcommand{\ignore}[1]{} %used to make inline comments
\definecolor{gray}{rgb}{0.1, 0.1, 0.1}
\newcommand{\gray}[1]{\colorbox{gray}{#1}}
\usepackage[utf8]{inputenc}
\usepackage[most]{tcolorbox}
\usepackage{hyperref}
\usepackage{listings}

\title{Lecture 2}
\author{Steven Walton\\     %Note that we don't close until after address to keep in title format.
\textit{PS 232 Computational Methods}\\
\textit{Department of Physics}\\
\textit{Embry-Riddle Aeronautical University}\\
\textit{Prescott, AZ   86301}}

\tcbset{
       frame code={}
       center title,
       left=0pt,
       right=0pt,
       top=0pt,
       bottom=0pt,
       colback=gray!70,
       colframe=white,
       width=\dimexpr\textwidth\relax,
       enlarge left by=0mm,
       boxsep=5pt,
       arc=0pt,outer arc=0pt,
       }

\begin{document}

\maketitle

\section*{Importing A Module}
Now that we have talked about modules, let's use them. One command that we are going to have to know is \gray{import}.
What this does is tells python that you care going to use the module specified.  So for example, let's create a random array.
\begin{tcolorbox}
   \begin{lstlisting}
   import numpy                  # We want to import the module numpy for 
                                 # the command random
   s = numpy.random.random(5)    # Take special note to how this is called.  
                                 # Do you have to do it this way?
   print s
   \end{lstlisting}
\end{tcolorbox}


You will see here that we are outputting the value of s, which is an array of 5 random numbers between 0 and 1.
Notice the tricky structure of how we called this this random command.  We had to use all those periods.  What this means
is that we are using the module NumPy, then using the library of Random, and the command random in it. This can be
tiresome to write over and over, especially if we only want to be calling random from NumPy and nothing else.  We have
a way to deal with this, and to take it one step further.  See if you can understand the following code.
\begin{tcolorbox}
   \begin{lstlisting}
   from numpy import random as ran
   s = ran.random(5)
   print s
   \end{lstlisting}
\end{tcolorbox}
So instead of importing all of numpy, we saved some time by just importing the \href{http://docs.scipy.org/doc/numpy/reference/routines.random.html}{random library}.  We also used ``as'' to 
shorten the name.  This can be a big help when we are using different modules.  Let's do a better example, we'll create 10 random points and plot them.  
\begin{tcolorbox}
   \begin{lstlisting}
   from numpy import random as ran
   import matplotlib.pyplt as plt      # Python's plotting module

   x = ran.random(5)                   # remember it's between (0,1)
   y = ran.random(5)

   plt.plot(x,y, 'ro')                 # Extremely similar to MATLAB's 
                                       # plotting function
   plt.show()
   \end{lstlisting}
\end{tcolorbox}
Now we see two ways to actually do the same thing, since pyplt is a library in matplotlib.  Now you may be asking what the third parameter of the plot command is ('ro').  The r means
that it will plot in red, and the o means that it will plot circles.  The default for plt is that it will connect the points with a line and the colour will be blue.  See the manual 
for other options, it is extremely similar to MATLAB's plotting function.

\section*{Types and Arrays}
One of the difficult things in programming is memory management.  In something like C++ you have to define things like character, float, long, boolean, or others.  Python does a lot of
this for you.  If we want to set a variable a to 100 we can do that by \gray{a=100}.  If we want to set a to a sentence \gray{a = ``We do it like this.''}.  For now we won't address
these issues because python will take care of most of this for you.
\\
We will address arrays though, since they will be more useful to us in a mathematical programming perspective.  Again, python will automatically take care of this type for you without
having to worry about memory management.  Let's make a list of five numbers, from 0 to 5.  Then what we want to do is output the number three.  We can do that quite easily.
\begin{tcolorbox}
   \begin{lstlisting}
   list = [0,1,2,3,4,5]    # Makes an array from 0-5
   print list[3]           # Prints 3
   \end{lstlisting}
\end{tcolorbox}

Pay careful attention to how this works, the listing starts at 0.  So \gray{list[0]} outputs 0, and \gray{list[5]} outputs 5.  It is important to remember that the ordering starts
from 0, and not 1.  Now let's say that we want to create a matrix.  For this we are going to create a 3x3 matrix, then output the second row, and the number in the third row and third 
column. Play with this and make sure that you are able to access every part of the matrix.  
\begin{tcolorbox}
   \begin{lstlisting}
   matrix = [[1,2,3],[4,5,6],[7,8,9]]  # 3x3 matrix
   print matrix[1]                     # prints second row
   print matrix[:][1]                  # also prints second row
   print matrix[2][2]                  # prints 9
   \end{lstlisting}
\end{tcolorbox}
For a more friendly looking version of this same thing we can use numpy's matrix command.  After importing it, there are two ways we can make the same exact matrix.
\begin{tcolorbox}
   \begin{lstlisting}
   np.matrix('1 2 3; 4 5 6; 7 8 9')
   np.matrix([[1,2,3],[4,5,6],[7,8,9]])
   \end{lstlisting}
\end{tcolorbox}
You'll notice that the output of these is much friendlier looking.  We can use the same structure to accessing the elements.  There are some key differences between these two functions.
One useful thing about the normal array in python is that we can append it and remove elements from it.  So let's add the number 1 to the end of our matrix, and then remove it.  (Calling
it a matrix is misleading, as you'll see.  But it is close, that we can think of it this way for now).
\begin{tcolorbox}
   \begin{lstlisting}
   matrix.append(1)     # [[ 1, 2, 3], [4, 5, 6], [7, 8, 9], 1]
   matrix.pop(3)        # removes 1 (4th element)
   \end{lstlisting}
\end{tcolorbox}
This is extremely useful if we need an array that needs to be updated.  We can not do this with the matrix from numpy (try it yourself).  What is nice about the matrix command is that 
we have all the same matrix commands available to us. (See \href{http://docs.scipy.org/doc/numpy/reference/generated/numpy.matrix.html}{numpy.matrix})
\\
There are two more commands that I'd like to introduce to you, which will help you with creating a range of numbers.  They are \gray{np.arange} and \gray{np.linspace}.  Linspace is the 
same as the linspace in Matlab.  It takes three parameters: starting point, ending point, and number of points.  So if we want to generate 1000 points between 0 and 10, we would use 
\gray{np.linspace(0,10,1000)}. This will store them into an array.  The next command is extremely similar. Arange also has three parameters but the third parameter you are specifying the
step size.  For example \gray{np.arange(0,10,2)} outputs [0,2,4,6,8].  One thing to keep in mind is that linspace will include the end point, where arange won't.  If you want to include
the final point in arange, remember to increase your range by one.  
\\
For good coding practices it is good to make the third parameter of these functions a variable.  This will help you a lot when you start using loops and need to keep these sizes the same.
It will also mean that you will only need to change one part of your code instead of multiple places.  Remember to keep code simple.  



\end{document}
