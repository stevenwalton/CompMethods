\documentclass[11pt]{article}   %Needed for every document
\usepackage[margin=0.75in]{geometry}           %Going to help get paper size right
\geometry{letterpaper}          %Normal paper
\usepackage{multicol}
\usepackage{graphicx,epsfig}    %Graphics packages and pictures
\usepackage{amssymb}            %Package for symbols

\usepackage{epstopdf}
\usepackage{color}
\graphicspath{ {Figures/} }
\DeclareGraphicsRule{.tif}{png}{.png}{`convert #1 `dirname #1`/`basename #1 .tif`.png}  %Need for pictures
\newcommand{\ignore}[1]{} %used to make inline comments
\definecolor{gray}{rgb}{0.1, 0.1, 0.1}
\newcommand{\gray}[1]{\colorbox{gray}{#1}}
\usepackage[utf8]{inputenc}
\usepackage[most]{tcolorbox}
\usepackage{hyperref}
\usepackage{listings}

\title{JSON & Webscraping}
\author{Steven Walton\\     %Note that we don't close until after address to keep in title format.
\textit{PS 232 Computational Methods}\\
\textit{Department of Physics}\\
\textit{Embry-Riddle Aeronautical University}\\
\textit{Prescott, AZ   86301}}

\tcbset{
       frame code={}
       center title,
       left=0pt,
       right=0pt,
       top=0pt,
       bottom=0pt,
       colback=gray!70,
       colframe=white,
       width=\dimexpr\textwidth\relax,
       enlarge left by=0mm,
       boxsep=5pt,
       arc=0pt,outer arc=0pt,
       }

\begin{document}

\maketitle
\section*{What is JSON}
JSON is short for JavaScript Object Notation.  It is an open standard format for human readable transmission of data objects.  JSON has six basic types; numbers, string, booleans, arrays, objects, and null.  To get a basic feel of 
what JSON looks like, here is an example:
\begin{tcolorbox}
   \begin{lstlisting}
   {
      ``firstName'': ``John'',
      ``lastName'': ``Smith'',
      ``isAlive'': true,
      ``age'': 25,
      ``height_cm'': 167.6,
      ``address'': {
         ``streetAddress'': ``21 Jump Street'',
         ``city'': ``New York'',
         ``state'': ``NY'',
         ``postalCode'': 10021-3100``
      },
      ''phoneNumbers``:[
         {
            ''type``: ''home``,
            ''number``: ''212 555-1234``
         },
         {
            ''type``: ''office``,
            ''number``: ''646 555-4567``
         }
      ],
      ''children``: [],
      ''spouse``: null
   }
   \end{lstlisting}
\end{tcolorbox}
If you remember back to when we talked about keywords and dictionaries, lecture 4, this will look extremely similar to you.
\\
Now you are probably asking why you should know this, and what's useful for.  Let's pretend that we want to get something off of the internet, text, data, whatever.  We use JSON.  The standard is used so that we can extract the needed
information to any language that we want.  We can also do the opposite and output data into JSON notation.  This is really great for doing things like *cough* outputting data from a telescope array to the internet *cough*. (Though that person might want to use a relational database and json to output the data to the user on the client side).
If you are writing to the web you should always be using JSON format, and if you are reading from the web you should know how JSON works.  A lot of sites use JSON, including wikipedia.  So if you are interacting with websites in 
any way, this is important to know. I will also suggest that if you are interested in websites and python that you look into the django module (the d is silent).

\section*{Some simple examples}
Let's just dive right in and try some simple output with json.  This is a great thing to do from a command line interpreter.  I suggest this for practice over using the compiler.
\begin{tcolorbox}
   \begin{lstlisting}
   import json    # I hope you know you should have to do this

   data = [ ( 'a': 'A', 'b': (2,4), 'c':3.0} ]
   print 'Inside the brackets is what the json type will look like:', 
          json.dumps(data, sort_keys=True, indent=2)
   \end{lstlisting}
\end{tcolorbox}
As you'll notice, this gives the same style as our json example above.  This is great if you are creating a website and need to have data read to the user.  I'm not going to go into how those APIs work, and just stick with python.
\\
Let's work with some Google APIs for the moment.
\subsection*{Google Maps}
So we have to carefully word things when using Google APIs.  Remember to use the maps.googleapis.com address.  So let's try a simple example.  We'll look at the address of the school and export some stuff.
\begin{tcolorbox}
   \begin{lstlisting}
   import pprint as pp
   import urllib2
   import json
   # Assume from now on that I am importing these

   url = ``https://maps.googleapis.com/maps/api/geocode/json?address=
            Embry-Riddle+Aeronautical+University+-+Prescott''
         # Good idea to write the url as a variable for easy access and
         # easy reading
   googleResponse = urllib2.urlopen(url)  # We want to see Google's response
   jsonResponse = json.loads(googleResponse.read())
   pp.pprint(jsonResponse)

   \end{lstlisting}
\end{tcolorbox}
Now we'll notice that the output gives us all the nice information about the campus.  Note that if you don't do pretty print that you'll get a messy output.  But from here we really have all the data that we need.  If we want to
get the longitude and latitude we can do this in two ways.  They are
\begin{tcolorbox}
   \begin{lstlisting}
   # Latitude and longitude in one go with json format still
   jsonResponse['results'][0]['geometry']['location']

   # If we want pure float numbers
   lat = jsonResponse['results'][0]['geometry']['location']['lat']
   lng = jsonResponse['results'][0]['geometry']['location']['lng']
   \end{lstlisting}
\end{tcolorbox}
Note the difference in outputs (if you aren't in command line remember to print these statements out).  But from this information we can actually do some cool things.  Let's create a program that will show us a map of any location we
want, by name.
\\\\
Again, I am not going to show you the entire code, but have included it on the GitHub page under this lecture folder.  You will be able to type any of address that Google recognizes.  Some examples you might want to try are:
\\\\
Embry-Riddle Aeronautical University - Prescott\\
Embry-Riddle Aeronautical University - Prescott King Engineering and Technology Center\\
Golden Gate Bridge\\
China\\
Space Needle\\
My Future\\
77 Massachusetts Ave Cambridge, MA 02139\\
The President\\
22.434335 146.203571\\\\
Have fun with the program.

\\
You can make the program a lot simpler if you just want to display the map.  See if you reduce it to do just that.  You should be able to do this in 10 
lines (not including the imports and error handling).  
\end{document}
